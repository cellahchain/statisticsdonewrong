%% Chapter 10 %%%
\chapter{Введение в анализ данных}
\label{chp10}

\section{Мощность \emph{p}-значений}
\label{chp10: }

Hiding the data

    “Given enough eyeballs, all bugs are shallow.”

    —Eric S. Raymond

We’ve talked about the common mistakes made by scientists, and how the best way to spot them is a bit of outside scrutiny. Peer review provides some of this scrutiny, but a peer reviewer doesn’t have the time to extensively re-analyze data and read code for typos – reviewers can only check that the methodology makes good sense. Sometimes they spot obvious errors, but subtle problems are usually missed.52

This is why many journals and professional societies require researchers to make their data available to other scientists on request. Full datasets are usually too large to print in the pages of a journal, so authors report their results and send the complete data to other scientists if they ask for a copy. Perhaps they will find an error or a pattern the original scientists missed.

Or so it goes in theory. In 2005, Jelte Wicherts and colleagues at the University of Amsterdam decided to analyze every recent article in several prominent journals of the American Psychological Association to learn about their statistical methods. They chose the APA partly because it requires authors to agree to share their data with other psychologists seeking to verify their claims.

Of the 249 studies they sought data for, they had only received data for 64 six months later. Almost three quarters of study authors never sent their data.61

Of course, scientists are busy people, and perhaps they simply didn’t have the time to compile their datasets, produce documents describing what each variable means and how it was measured, and so on.

Wicherts and his colleagues decided they’d test this. They trawled through all the studies looking for common errors which could be spotted by reading the paper, such as inconsistent statistical results, misuse of various statistical tests, and ordinary typos. At least half of the papers had an error, usually minor, but 15% reported at least one statistically significant result which was only significant because of an error.

Next, they looked for a correlation between these errors and an unwillingness to share data. There was a clear relationship. Authors who refused to share their data were more likely to have committed an error in their paper, and their statistical evidence tended to be weaker.60 Because most authors refused to share their data, Wicherts could not dig for deeper statistical errors, and many more may be lurking.

This is certainly not proof that authors hid their data out of fear their errors may be uncovered, or even that the authors knew about the errors at all. Correlation doesn’t imply causation, but it does waggle its eyebrows suggestively and gesture furtively while mouthing “look over there.”[1]
Just leave out the details

Nitpicking statisticians getting you down by pointing out flaws in your paper? There’s one clear solution: don’t publish as much detail! They can’t find the errors if you don’t say how you evaluated your data.

I don’t mean to seriously suggest that evil scientists do this intentionally, although perhaps some do. More frequently, details are left out because authors simply forgot to include them, or because journal space limits force their omission.

It’s possible to evaluate studies to see what they left out. Scientists leading medical trials are required to provide detailed study plans to ethical review boards before starting a trial, so one group of researchers obtained a collection of these plans from a review board. The plans specify which outcomes the study will measure: for instance, a study might monitor various symptoms to see if any are influenced by the treatment. The researchers then found the published results of these studies and looked for how well these outcomes were reported.

Roughly half of the outcomes never appeared in the scientific journal papers at all. Many of these were statistically insignificant results which were swept under the rug.[2] Another large chunk of results were not reported in sufficient detail for scientists to use the results for further meta-analysis.14

Other reviews have found similar problems. A review of medical trials found that most studies omit important methodological details, such as stopping rules and power calculations, with studies in small specialist journals faring worse than those in large general medicine journals.29

Medical journals have begun to combat this problem with standards for reporting of results, such as the CONSORT checklist. Authors are required to follow the checklist’s requirements before submitting their studies, and editors check to make sure all relevant details are included. The checklist seems to work; studies published in journals which follow the guidelines tend to report more essential detail, although not all of it.46 Unfortunately the standards are inconsistently applied and studies often slip through with missing details nonetheless.42 Journal editors will need to make a greater effort to enforce reporting standards.

We see that published papers aren’t faring very well. What about unpublished studies?
Science in a filing cabinet

Earlier we saw the impact of multiple comparisons and truth inflation on study results. These problems arise when studies make numerous comparisons with low statistical power, giving a high rate of false positives and inflated estimates of effect sizes, and they appear everywhere in published research.

But not every study is published. We only ever see a fraction of medical research, for instance, because few scientists bother publishing “We tried this medicine and it didn’t seem to work.”

Consider an example: studies of the tumor suppressor protein TP53 and its effect on head and neck cancer. A number of studies suggested that measurements of TP53 could be used to predict cancer mortality rates, since it serves to regulate cell growth and development and hence must function correctly to prevent cancer. When all 18 published studies on TP53 and cancer were analyzed together, the result was a highly statistically significant correlation: TP53 could clearly be measured to tell how likely a tumor is to kill you.

But then suppose we dig up unpublished results on TP53: data that had been mentioned in other studies but not published or analyzed. Add this data to the mix and the statistically significant effect vanishes.36 After all, few authors bothered to publish data showing no correlation, so the meta-analysis could only use a biased sample.

A similar study looked at reboxetine, an antidepressant sold by Pfizer. Several published studies have suggested that it is effective compared to placebo, leading several European countries to approve it for prescription to depressed patients. The German Institute for Quality and Efficiency in Health Care, responsible for assessing medical treatments, managed to get unpublished trial data from Pfizer – three times more data than had ever been published – and carefully analyzed it. The result: reboxetine is not effective. Pfizer had only convinced the public that it’s effective by neglecting to mention the studies proving it isn’t.18

This problem is commonly known as publication bias or the file-drawer problem: many studies sit in a file drawer for years, never published, despite the valuable data they could contribute.

The problem isn’t simply the bias on published results. Unpublished studies lead to a duplication of effort – if other scientists don’t know you’ve done a study, they may well do it again, wasting money and effort.

Regulators and scientific journals have attempted to halt this problem. The Food and Drug Administration requires certain kinds of clinical trials to be registered through their website ClinicalTrials.gov before the trials begin, and requires the publication of results within a year of the end of the trial. Similarly, the International Committee of Medical Journal Editors announced in 2005 that they would not publish studies which had not been pre-registered.

Unfortunately, a review of 738 registered clinical trials found that only 22% met the legal requirement to publish.47 The FDA has not fined any drug companies for noncompliance, and journals have not consistently enforced the requirement to register trials. Most studies simply vanish.
[1]	Joke shamelessly stolen from the alternate text of http://xkcd.com/552/.
[2]	Why do we always say “swept under the rug”? Whose rug is it? And why don’t they use a vacuum cleaner instead of a broom?