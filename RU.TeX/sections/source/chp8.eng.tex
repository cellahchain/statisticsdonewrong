%% Chapter 8 %%%
\chapter{Researcher freedom: good vibrations?}
\label{chp8}

There’s a common misconception that statistics is boring and monotonous. Collect lots of data, plug the numbers into Excel or SPSS or R, and beat the software with a stick until it produces some colorful charts and graphs. Done! All the statistician must do is read off the results.

But one must choose which commands to use. Two researchers attempting to answer the same question may perform different statistical analyses entirely. There are many decisions to make:

    Which variables do I adjust for? In a medical trial, for instance, you might control for patient age, gender, weight, BMI, previous medical history, smoking, drug use, or for the results of medical tests done before the start of the study. Which of these factors are important, and which can be ignored?
    Which cases do I exclude? If I’m testing diet plans, maybe I want to exclude test subjects who came down with uncontrollable diarrhea during the trial, since their results will be abnormal.
    What do I do with outliers? There will always be some results which are out of the ordinary, for reasons known or unknown, and I may want to exclude them or analyze them specially. Which cases count as outliers, and what do I do with them?
    How do I define groups? For example, I may want to split patients into “overweight”, “normal”, and “underweight” groups. Where do I draw the lines? What do I do with a muscular bodybuilder whose BMI is in the “overweight” range?
    What about missing data? Perhaps I’m testing cancer remission rates with a new drug. I run the trial for five years, but some patients will have tumors reappear after six years, or eight years. My data does not include their recurrence. How do I account for this when measuring the effectiveness of the drug?
    How much data should I collect? Should I stop when I have a definitive result, or continue as planned until I’ve collected all the data?
    How do I measure my outcomes? A medication could be evaluated with subjective patient surveys, medical test results, prevalence of a certain symptom, or measures such as duration of illness.

Producing results can take hours of exploration and analysis to see which procedures are most appropriate. Papers usually explain the statistical analysis performed, but don’t always explain why the researchers chose one method over another, or explain what the results would be had the researchers chosen a different method. Researchers are free to choose whatever methods they feel appropriate – and while they may make the right choices, what would happen if they analyzed the data differently?

In simulations, it’s possible to get effect sizes different by a factor of two simply by adjusting for different variables, excluding different sets of cases, and handling outliers differently.30 The effect size is that all-important number which tells you how much of a difference your medication makes. So apparently, being free to analyze how you want gives you enormous control over your results!

The most concerning consequence of this statistical freedom is that researchers may choose the statistical analysis most favorable to them, arbitrarily producing statistically significant results by playing with the data until something emerges. Simulation suggests that false positive rates can jump to over 50% for a given dataset just by letting researchers try different statistical analyses until one works.53

Medical researchers have devised ways of preventing this. Researchers are often required to draft a clinical trial protocol, explaining how the data will be collected and analyzed. Since the protocol is drafted before the researchers see any data, they can’t possibly craft their analysis to be most favorable to them. Unfortunately, many studies depart from their protocols and perform different analysis, allowing for researcher bias to creep in.15, 14 Many other scientific fields have no protocol publication requirement at all.

The proliferation of statistical techniques has given us many useful tools, but it seems they have been put to use as blunt objects. One must simply beat the data until it confesses.