
An introduction to data analysis

Much of experimental science comes down to measuring changes. Does one medicine work better than another? Do cells with one version of a gene synthesize more of an enzyme than cells with another version? Does one kind of signal processing algorithm detect pulsars better than another? Is one catalyst more effective at speeding a chemical reaction than another?

Much of statistics, then, comes down to making judgments about these kinds of differences. We talk about “statistically significant differences” because statisticians have devised ways of telling if the difference between two measurements is really big enough to ascribe to anything but chance.

Suppose you’re testing cold medicines. Your new medicine promises to cut the duration of cold symptoms by a day. To prove this, you find twenty patients with colds and give half of them your new medicine and half a placebo. Then you track the length of their colds and find out what the average cold length was with and without the medicine.

But all colds aren’t identical. Perhaps the average cold lasts a week, but some last only a few days, and others drag on for two weeks or more, straining the household Kleenex supply. It’s possible that the group of ten patients receiving genuine medicine will be the unlucky types to get two-week colds, and so you’ll falsely conclude that the medicine makes things worse. How can you tell if you’ve proven your medicine works, rather than just proving that some patients are unlucky?


The power of p values

Statistics provides the answer. If we know the distribution of typical cold cases – roughly how many patients tend to have short colds, or long colds, or average colds – we can tell how likely it is for a random sample of cold patients to have cold lengths all shorter than average, or longer than average, or exactly average. By performing a statistical test, we can answer the question “If my medication were completely ineffective, what are the chances I’d see data like what I saw?”

That’s a bit tricky, so read it again.

Intuitively, we can see how this might work. If I only test the medication on one person, it’s unsurprising if he has a shorter cold than average – about half of patients have colds shorter than average. If I test the medication on ten million patients, it’s pretty damn unlikely that all of them will have shorter colds than average, unless my medication works.

The common statistical tests used by scientists produce a number called the p value that quantifies this. Here’s how it’s defined:

    The P value is defined as the probability, under the assumption of no effect or no difference (the null hypothesis), of obtaining a result equal to or more extreme than what was actually observed.24

So if I give my medication to 100 patients and find that their colds are a day shorter on average, the p value of this result is the chance that, if my medication didn’t do anything at all, my 100 patients would randomly have day-shorter colds. Obviously, the p value depends on the size of the effect – colds shorter by four days are less likely than colds shorter by one day – and the number of patients I test the medication on.

That’s a tricky concept to wrap your head around. A p value is not a measure of how right you are, or how significant the difference is; it’s a measure of how surprised you should be if there is no actual difference between the groups, but you got data suggesting there is. A bigger difference, or one backed up by more data, suggests more surprise and a smaller p value.

It’s not easy to translate that into an answer to the question “is there really a difference?” Most scientists use a simple rule of thumb: if p is less than 0.05, there’s only a 5\% chance of obtaining this data unless the medication really works, so we will call the difference between medication and placebo “significant.” If p is larger, we’ll call the difference insignificant.

But there are limitations. The p value is a measure of surprise, not a measure of the size of the effect. I can get a tiny p value by either measuring a huge effect – “this medicine makes people live four times longer” – or by measuring a tiny effect with great certainty. Statistical significance does not mean your result has any practical significance.

Similarly, statistical insignificance is hard to interpret. I could have a perfectly good medicine, but if I test it on ten people, I’d be hard-pressed to tell the difference between a real improvement in the patients and plain good luck. Alternately, I might test it on thousands of people, but the medication only shortens colds by three minutes, and so I’m simply incapable of detecting the difference. A statistically insignificant difference does not mean there is no difference at all.

There’s no mathematical tool to tell you if your hypothesis is true; you can only see whether it is consistent with the data, and if the data is sparse or unclear, your conclusions are uncertain.

But we can’t let that stop us.
