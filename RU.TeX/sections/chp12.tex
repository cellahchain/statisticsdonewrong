%% Chapter 12 %%%
\chapter{Что можно сделать?}
\label{chp12}

На протяжении этой книги мы обсуждали большое количество статистических проблем, которые возникают во многих областях науки: медицине, физике, климатических исследованиях, нейронауках, и многих других. Любой исследователь, использующий статистические методы для анализа данных, может допустить ошибку, и, как мы уже видели, многие допускают. Что мы можем с этим сделать?


\section{Статистическое обучение}
\label{chp12:statisticaleducation}

Большинство американских студентов имеют лишь минимальное статистическое образование - один или два обязательных курса, а у многих вообще ни одного. И даже если студенты прошли обучение по курсу, преподаватели отмечают, что они не способны применять статистические понятия к научным вопросам, поскольку никогда не понимали или просто быстро забывали соответствующие методы. Это необходимо изменить. Практически каждая научная дисциплина зависит от статистического анализа экспериментальных данных, а статистические ошибки просто обесценивают время исследователей и финансовые гранты. 

В некоторых университетах экспериментировали, пытаясь совмещать статистические курсы и научные занятия, чтобы студенты могли сразу применять свои статистические знания к проблемам в изучаемых научных сферах. Предварительные результаты показывают, что это работает: студенты узнают и запоминают больше статистических методов, и меньше жалуются на то, что их принуждают изучать статистику.\cite{metz_teaching_2008} Другие университеты должны перенять такие методы, используя концептуальные тесты, чтобы определить, какие методы работают лучше всего.  

Учебные материалы должны быть также более доступными. Я познакомился со статистикой, когда мне нужно было проанализировать данные, полученные в лаборатории, и я не знал как это сделать; пока статистическое обучение не станет повсеместным, многие студенты будут оказываться в схожих ситуациях - и им нужны будут источники информации. Такие проекты, как \href{https://www.openintro.org/stat/textbook.php}{OpenIntro Stats}, выглядят многообещающими, и, я надеюсь, в ближайшем будущем их станет еще больше.


\section{Научные публикации}
\label{chp12:sciencepublishing}

Научные журналы медленно добиваются прогресса в решении тех проблем, которые я обсуждал. Такие принципы, как CONSORT для рандомизированных испытаний, делают ясным, какая информация требуется должна быть в публикуемой статье, чтобы она была воспроизводимой. К сожалению, как мы уже видели ранее, эти принципы предписываются нечасто. Мы должны продолжать оказывать давление на журналы, чтобы они требовали от своих авторов соблюдения строгих стандартов.

Главные журналы должны возглавить эту тенденцию. Журнал \emph{Nature} уже так сделал, объявив новый \href{http://www.nature.com/authors/policies/checklist.pdf}{перечень требований}, которому авторы должны полностью следовать, чтобы их статьи были опубликованы. В этот перечень будут входить требования публикации размеров выборки, подсчета статистической мощности, регистрационные номера клинических испытаний, полный список требований CONSORT, корректировка для множественных сравнений и обмен данными и исходным кодом. Этот перечень покрывает практически все проблемы, рассмотренные в этой книге, за исключением \hyperref[chp7]{правил остановки} и обсуждения причин отклонения от зарегистрированного \hyperref[chp8]{протокола} исследования. \emph{Nature} также сделает доступными статистические консультации для статей, если потребуется. 

Если эти принципы сделать законом, тогда в результате мы получим более надежные и воспроизводимые научные исследования. Другие журналы должны делать тоже самое.


\section{Ваша задача}
\label{chp12:yourjob}

Ваша задача может быть выражена в виде четырёх простых шагов:

\begin{enumerate}

	\item Прочитайте хорошую книгу по статистике или пройдите хороший курс. Практикуйтесь.
	\item Планируйтесвой анализ данных тщательно и аккуратно, стараясь избегать заблуждений и ошибок, о которых вы узнали.
	\item Если вы обнаружили знакомую ошибку в научной литературе, такую как, например, неправильную интерпретацию \emph{p-}значений, - просто ударьте виновного по голове своим учебником по статистике. Это поможет. % или "В терапевтических целях".
	\item Добивайтесь изменений в научном образовании и публикациях. Это наше исследование. Давайте не облажаемся.

\end{enumerate}


%% Chapter 13: Conclusion %%%
\chapter*{Заключение}
\label{chp13}
\addcontentsline{toc}{chapter}{Заключение}

Остерегайтесь ложной уверенности. У вас может довольно быстро появиться самодовольное чувство удовлетворения от того, что ваша работа всегда идеальна, в отличие от других. Но в этой книге не было полного введения в математику, стоящую за анализом данных. Есть много других способов испортить статистику и за пределами этих простых концептуальных ошибок.

Ошибки будут происходить часто, поскольку лишь некоторые научные бакалаврские программы или медицинские школы требуют изучения курсов по статистике и дизайну экспериментов, - и некоторые вводные курсы по статистике опускают рассмотрение вопросов статистической мощности и множественных следствий. Это выглядит достаточным, несмотря на первостепенную роль данных и статистического анализа в стремлениях и поисках современной науки; мы ведь не принимаем докторов, у которых нет опыта работы с рецептурными лекарствами, так почему мы принимаем ученых, не имеющих статистической подготовки? Ученым нужно формальное статистическое обучение и консультации. Цитата:\\

%\newline

\begin{chapquote}{Р.А. Фишер, популяризатор \emph{p}-значения}
``Проконсультироваться с статистиком после того, как эксперимент был закончен, - это как попросить его провести посмертное вскрытие. Он, возможно, скажет, из-за чего умер эксперимент.''
\end{chapquote} 

Журналы могут отказывать в публикации исследованиям с статистическим анализом низкого качества, а новые требования и протоколы могут устранить некоторые проблемы, но никакого улучшения в планировании экспериментов и анализе данных не будет до тох пор, пока у нас не будет ученых, обученных в соответствии с принципами статистики. Лишь продолжатся всепоглощающие поиски статистической значимости.

Изменения не будут простыми. Строгие статистические стандарты не даются даром: например, если ученые начнут регулярно делать подсчеты статистической мощности, скоро обнаружится, что им нужны значительно б\emph{о}льшие размеры выборок, чтобы достичь убедительных выводов. Клинические испытания не бесплатны, и более дорогая стоимость исследований означает меньшее количество опубликованных испытаний. Вы можете возразить, что научный прогресс будет излишне замедлен - но разве не хуже строить наш прогресс на фундаменте необоснованных результатов?

Для студентов, изучающих науку: вкладывайте в статистические курсы, пока у вас есть возможность. Исследователям: вкладывайте в обучение, хорошую книгу и полезный статистический совет. И когда в следующий раз вы услышите, как кто-то говорит: ``Результат оказался значимым на уровне $p<0,05$, т.е. есть лишь 1 шанс из 20, что это случайность!'', - пожалуйста, стукните его по голове учебником по статистике за меня.

\textbf{Отказ от ответственности}: Советы, приведённые в этом руководстве, не могут заменить консультации квалифицированного специалиста по статистике. Если вам кажется, что вы страдаете от любой серьезной статистической ошибки, пожалуйста, немедленно проконсультируйтесь со статистиком. Я не несу никакой ответственности в случае, если в результате использования данного веб-сайта и руководства пострадало ваше достоинство.

Использование этого руководства в целях оправдания отрицания результатов научного исследования без подробного рассмотрения любых доказательств будет основанием получить сверху по голове очень большим учебником по статистике. Это руководство должно помочь вам находить статистические ошибки, а не позволить вам выборочно игнорировать те части науки, которые вам не нравятся.

