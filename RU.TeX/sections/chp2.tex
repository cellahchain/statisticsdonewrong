%% Section 2 %%% Введение в анализ данных


\chapter{Введение в анализ данных}
\label{chp2}

Большая часть экспериментальной науки сводится к измерению каких-либо изменений. Действует ли одно лекарство лучше другого? Синтезируют ли клетки с одним набором генов большее количество ферментов, чем клетки с другим набором? Какой из алгоритмов обработки сигнала лучше обнаруживает пульсары? Является ли один катализатор эффективнее другого?

Таким образом, большинство статистических расчётов сводится к принятию решения о том, существуют ли различия между измерениями. Мы говорим о т.н. "статистически значимых различиях"\hspace{1pt}, поскольку статистики изобрели методы, позволяющие определять ситуации, когда различия между несколькими измерениями значимы - т.е. невозможно утверждать, что эти различия возникли случайным образом.

Допустим, вы хотите протестировать эффективность средств от простуды. Некоторое новое лекарство, предположительно, сокращает продолжительность проявления симптомов простуды на 1 день. Чтобы проверить данное утверждение, вы находите 20 пациентов со схожими симптомами, делите их на равные группы; одной группе предлагаете новое лекарство, а другой - плацебо. Далее вы отслеживаете продолжительность проявления симптомов простуды у каждой группы и подсчитываете ее среднее значение для каждой из групп.

Но не все простуды одинаковы. Возможно, простуда у человека длится около недели, в среднем, но у некоторых она может продолжаться всего несколько дней, в то время как у других - 2 недели и более, серьезно истощая домашние запасы салфеток. Вполне вероятно, что та группа из 10 человек, которая получала лекарство, могла состоять как раз из таких счастливчиков с двухнедельной простудой, и тогда вы сделаете ошибочный вывод о том, что лекарство только ухудшает ситуацию.  Как определить, что мы доказали эффективность лекарства, а не то, что мы выбрали неудачную выборку для эксперимента?

\section{Мощность \emph{p}-значений}
\label{chp2:pvalues}

Статистика дает ответ на этот вопрос. Если нам известно \emph{распределение} случаев типичной простуды - примерное количество пациентов с симптомами краткосрочной простуды, длительной и средней по длительности простуды - мы можем оценить насколько вероятно то, что некая случайная выборка пациентов с простудой будет состоять из людей с продолжительностью простуды меньше, чем в среднем, больше, чем в среднем или идентично средней длительности. Используя статистический тест, мы можем ответить на вопрос: ``Если мое лекарство не подействовало, какова вероятность того, что я увижу такие же данные, что я уже видел?''
Звучит несколько сложно, поэтому, прочтите ещё раз.


Интуитивно, мы можем увидеть, как это должно работать. Если я проверю лекарство только на 1 человеке, вполне ожидаемо, если он вылечится быстрее, чем в среднем длится простуда, - просто потому, что у половины пациентов простуда длится меньше, чем в среднем у всех пациентов. Однако, если проверить лекарство на 10 миллионах пациентов, чертовски маловероятно, что \emph{все} они вылечатся быстрее, чем в среднем, \emph{если только моё лекарство не подействовало}.    

Общепринятые статистические тесты, используемые статистиками, выводят число, называемое \emph{p}-значением, которое это подсчитывает. Вот как оно определяется:

\begin{quotation}

\textbf{\emph{P}-значение}\footnote{Альтернативное определение из \href{http://goo.gl/JHujIw}{Википедии}: \textit{\textbf{p-значение} равно вероятности того, что случайная величина с данным распределением (распределением тестовой статистики при нулевой гипотезе) примет значение, не меньшее, чем фактическое значение тестовой статистики}.} можно определить как вероятность получить результат равный или больший, чем в реальности наблюдаемый, при условии, что никакого эффекта или различий обнаружено не было (нулевая гипотеза)\cite{goodman_toward_1999}.   

\end{quotation}


Таким образом, если я дам свое лекарство 100 пациентам и увижу, что их простуды длятся на 1 день меньше, чем в среднем, р-значением такого результата будет вероятность того, что все мои 100 пациентов совершенно случайно имели простуду на 1 день короче, при условии, что моё лекарство не подействовало. Очевидно, что \emph{р}-значение зависит от размера эффекта (простуды, которые длятся на 4 дня меньше встречаются реже и менее вероятны, чем простуды, длящиеся на лишь 1 день меньше среднего значения) и от количества пациентов, на которых проверяется лекарство  


Это хитрое понятие, которое может запутать. \emph{P}-значение - не мера того, насколько вы правы или насколько значимы различия; скорее, оно измеряет то, \emph{насколько сильно вы должны быть удивлены} в случае, если никаких реальных различий между группами обнаружено не будет, а у вас есть данные, предполагающие обратное. Чем больше различия или чем больше данныих, подкрепляюших эти различия, тем больше степень удивления и меньше \emph{р}-значение.

Не так уж просто превратить это в ответ на вопрос: “а существуют ли в реальности различия?” Большинство ученых используют простое эмпирическое правило: если \emph{р}-значение меньше 0,05, существует только 5\% шанс получения такого результата, когда лекарство действительно не работает, - поэтому мы станем считать различия между лекарством и плацебо ``значимыми''. Если \emph{р}-значение больше, различия будут считаться незначимыми.

Но здесь существуют ограничения. \emph{Р}-значение - это мера нашей степени удивления, а не размер эффекта. Я могу получить крошечное р-значение либо измеряя огромный эффект - ``наше лекарство увеличивает продолжительность жизни людей в 4 раза'', либо измеряя очень маленький эффект с большой достоверностью. Статистическая значимость не означает, что ваш результат имеет какую-либо \emph{практическую} или \emph{фактическую} значимость.

Аналогичным образом сложно интерпретировать и статистическую \emph{не}значимость. У меня может быть прекрасное лекарство, но если я протестирую его всего на 10 пациентах, мне будет очень трудно определить, откуда взялись различия: вследствие реального действия лекарства или моей удачи. Или я мог бы попеременно протестировать лекарство на тысячах пациентов, а лекарство укорачивало бы простуду всего на 3 минуты, поэтому я просто был бы не в состоянии отследить такие различия. Статистически незначимое различие не означает, что в реальности различий не существует.

К сожалению, нет такого математического инструмента, который позволил бы определить, верна ли ваша гипотеза; вы можете только проверить, согласуется ли эта гипотеза с данными - и если данные беспорядочны или маловразумительны, ваши выводы будут такими же. 

Но нас это не остановит.