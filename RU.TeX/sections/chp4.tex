%% Chapter 4 %%%
\chapter{Псевдорепликация: выбирайте данные корректно}
\label{chp4}

Многие исследования стараются собрать больше данных путем репликации: повторяя свои измерения на дополнительных пациентах или выборках, они стремятся к подтвердить свои результаты и увидеть едва заметные взаимосвязи, которые не так очевидны на первый взгляд. Мы уже видели, насколько ценными могут быть дополнительные данные для улучшения статистической мощности и обнаружения небольших различий. Но что именно можно считать репликацией?

Воспользуемся снова медицинским примером. У меня есть 2 группы пациентов общим количеством в 100 человек, принимающие разные лекарства, и я пытаюсь установить, какое из лекарств лучше понижает кровяное давление. Чтобы увидеть результат, каждая из групп принимала лекарства в течении месяца, а потом я наблюдал за каждой из групп в течении 10 дней, ежедневно измеряя их давление. Таким образом, у меня есть по 10 переменных на каждого пациента и 1000 переменных на каждуюу группу. 

Замечательно! 1000 переменных - это достаточно много данных, и я могу относительно просто установить, уменьшилось ли давление у одной группы, по сравнению с другой. Если посчитать статистическую значимость различий - они определенно будут значимыми.

Однако, мы считали, что измеряя 10 раз давление у пациента, мы должны получить десять примерно одинаковых результатов. Если один из пациентов генетически предрасположен на низкое давление, я 10 раз посчитал его предрасположенность. Если бы я собрал данные от 1000 независимых пациентов вместо последовательного измерения 100 пациентов, я мог быть более уверен в том, что различия между группами являются следствием действия лекарства, а не генетики и случайной удачи. Я заявляю о большом размере моей выборки, которая дает мне статистически значимые результаты и высокую статистическую мощность, но это заявление неоправданно.  

Эта проблема часто встречается и известна под названием псевдорепликация\cite{lazic_problem_2010}. Протестировав несколько клеток одного микроорганизма, биолог может ``реплицировать'' свои результаты путем тестирования большего количества клеток из того же микроорганизма. Нейроучёные могут исследовать большое количество нейронов одного и того же животного, заявляя ошибочно, что у них была огромная выборка, потому что они исследовали несколько сотен нейронов из всего двух крыс.

В терминах статистики, псеворепликация происходит тогда, когда индивидуальные наблюдения сильно зависят друг от друга. Например, ваши измерения кровяного давления пациента будут тесно связаны с предыдущими измерениями, а результаты изучения состава почвы в одном месте будут положительно коррелировать с составом почвы в полуторах метрах от этого места. Существует несколько способов учета этой зависимости в процессе статистического анализа:

\begin{enumerate}
	\item Усреднение зависимых переменных. Например, можно усреднить все измерения кровяного давления у одного пациента, хотя это и не идеальный способ: если вы измеряли одних пациентов чаще, чем других, - это не отразится на среднем значении. Вам нужен будет метод, который каким-то образом будет учитывать следующее: чем больше измерений - тем более они надежны.
	\item Анализ каждой зависимой переменной по-отдельности. Можно использовать для анализа показания кровяного давления каждого пациента на пятый день, что даст только одну переменную на каждого пациента. Но здесь нужно быть осторожным, потому что если сделать это для каждого дня тестирования, могут возникнуть проблемы с \hyperref[chp5:try-again]{множественными сравнениями}, которые мы будем обсуждать в следущей главе.
	\item Использование статистических моделей, которые учитывают зависимые переменные, например, иерархическая модель данных или модель случайных эффектов.
\end{enumerate}  


Важно рассмотреть каждый из возможных подходов прежде, чем анализировать свои данные, поскольку каждый из методов лучше подходит к разным ситуациям. Псевдорепликация позволяет достичь статистической значимости довольно просто, хотя и не даёт никакой дополнительной информации об испытуемых. Исследователи должны быть аккуратны в своих выводах, не увеличивая искусственно размеры своей выборки путем повторного тестирования одной и той же выборки.