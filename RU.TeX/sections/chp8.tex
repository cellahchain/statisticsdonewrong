%% Chapter 8 %%%
\chapter{Свобода исследователя: хорошие сомнения? }
\label{chp8}

Существует распространенное заблуждение о том, что статистика скучна и однообразна. Соберите как можно больше данных, загрузите цифры в Excel, SPSS или R, и пинайте программу до тех пор, пока она не выдаст какие-нибудь красочные графики или диаграммы. Готово! Все, что остается статистикам, - это считывать результаты.

Однако, придётся выбирать, какие методы и команды использовать. Два исследователя, ищущих ответы на один и тот же вопрос, могут использовать совершенно разные методы статистического анализа. Нужно принимать множество решений:


\begin{enumerate}
	\item Какие переменные мне нужно учитывать? В медицинских испытаниях, например, стоит контролировать возраст пациентов, пол, вест, ИМТ, предыдущую медицинскую историю, курение, использование наркотиков или результаты медицинских тестов, проведённых до начала исследования. Какие из этих факторов важны, а какие можно игнорировать?

	\item Какие случаи можно исключить? Если я тестирую планы питания, возможно, я хочу исключить испытуемых, у которых в процессе исследования появилась неконтроллируемая диарея, так как их результаты будут выходить за рамки нормы.

	\item Что делать с выбросами? В    



\end{enumerate}

    
    
    What do I do with outliers? There will always be some results which are out of the ordinary, for reasons known or unknown, and I may want to exclude them or analyze them specially. Which cases count as outliers, and what do I do with them?
    
    How do I define groups? For example, I may want to split patients into “overweight”, “normal”, and “underweight” groups. Where do I draw the lines? What do I do with a muscular bodybuilder whose BMI is in the “overweight” range?
    
    What about missing data? Perhaps I’m testing cancer remission rates with a new drug. I run the trial for five years, but some patients will have tumors reappear after six years, or eight years. My data does not include their recurrence. How do I account for this when measuring the effectiveness of the drug?
    
    How much data should I collect? Should I stop when I have a definitive result, or continue as planned until I’ve collected all the data?
    
    How do I measure my outcomes? A medication could be evaluated with subjective patient surveys, medical test results, prevalence of a certain symptom, or measures such as duration of illness.

Producing results can take hours of exploration and analysis to see which procedures are most appropriate. Papers usually explain the statistical analysis performed, but don’t always explain why the researchers chose one method over another, or explain what the results would be had the researchers chosen a different method. Researchers are free to choose whatever methods they feel appropriate – and while they may make the right choices, what would happen if they analyzed the data differently?

In simulations, it’s possible to get effect sizes different by a factor of two simply by adjusting for different variables, excluding different sets of cases, and handling outliers differently.30 The effect size is that all-important number which tells you how much of a difference your medication makes. So apparently, being free to analyze how you want gives you enormous control over your results!

The most concerning consequence of this statistical freedom is that researchers may choose the statistical analysis most favorable to them, arbitrarily producing statistically significant results by playing with the data until something emerges. Simulation suggests that false positive rates can jump to over 50% for a given dataset just by letting researchers try different statistical analyses until one works.53

Medical researchers have devised ways of preventing this. Researchers are often required to draft a clinical trial protocol, explaining how the data will be collected and analyzed. Since the protocol is drafted before the researchers see any data, they can’t possibly craft their analysis to be most favorable to them. Unfortunately, many studies depart from their protocols and perform different analysis, allowing for researcher bias to creep in.15, 14 Many other scientific fields have no protocol publication requirement at all.

The proliferation of statistical techniques has given us many useful tools, but it seems they have been put to use as blunt objects. One must simply beat the data until it confesses.