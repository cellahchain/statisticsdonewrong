%% Chapter 8 %%%
\chapter{Свобода исследователя: хорошие сомнения? }
\label{chp8}

Существует распространенное заблуждение о том, что статистика скучна и однообразна. Соберите как можно больше данных, загрузите цифры в Excel, SPSS или R, и пинайте программу до тех пор, пока она не выдаст какие-нибудь красочные графики или диаграммы. Готово! Все, что остается статистикам, - это считывать результаты.

Однако, придётся выбирать, какие методы и команды использовать. Два исследователя, ищущих ответы на один и тот же вопрос, могут использовать совершенно разные методы статистического анализа. Нужно принимать множество решений:


\begin{enumerate}
	\item Какие переменные мне нужно учитывать? В медицинских испытаниях, например, стоит контролировать возраст пациентов, пол, вест, ИМТ, предыдущую медицинскую историю, курение, использование наркотиков или результаты медицинских тестов, проведённых до начала исследования. Какие из этих факторов важны, а какие можно игнорировать?

	\item Какие случаи можно исключить? Если я тестирую планы питания, возможно, я хочу исключить испытуемых, у которых в процессе исследования появилась неконтроллируемая диарея, так как их результаты будут выходить за рамки нормы.

	\item Что делать с выбросами? По неизвестным причинам, среди результатов всегда будут некоторые выпадающие из нормы, и, возможно, мне придется их исключить или анализировать отдельно. Какие случаи тогда можно считать выбросами и что мне с ними делать?

	\item Как мне определить группы? Например, я, возможно, хочу разделить испытуемых на группы ``нормальных'', ``с избыточным весом'' и ``с недостаточным весом''. Как определить границы групп? Что мне делать с мускулистыми бодибилдером, чей ИМТ находится в диапазоне значений группы ``с избыточным весом''?

	\item Что насчёт отсутствующих данных? Допустим, я пытаюсь оценить уменьшение распространения рака путем тестирования нового лекарства. Я провожу испытания в течении пяти лет, но у некоторых пациентов опухоли возвращаются через шесть или восемь лет, а в мои данные этого не будут содержать. Как мне это учитывать при оценке эффективности лекарства? 

	\item Сколько данных я должен собрать? Стоит ли мне остановиться, как только я получу определённый результат, или следует продолжать до тех пор, пока я не соберу все данные, как было запланировано?

	\item Как мне оценить полученные результаты? Действие лекарства можно оценить с помощью индивидуальных опросов пациентов, результатов испытаний, преобладания определенного симптома или, например, продолжительностью болезни.

\end{enumerate}

Прежде, чем появятся результаты, можно потратить много времени на определение того, какие методы будут наиболее уместны в данной ситуации. Научные статьи обычно содержат объяснение того, какие методы статистической обработки были использованы, но они практически никогда не объясняют причину выбора именно этого метода по сравнению с другими или какие результаты могли получить исследователи, используя другие методы. Исследователи вольны выбирать методы, которые они считают подходящими, - и, хотя они могли сделать правильный выбор, что было бы, анализируй они эти данные по-другому?    



In simulations, it’s possible to get effect sizes different by a factor of two simply by adjusting for different variables, excluding different sets of cases, and handling outliers differently.30 The effect size is that all-important number which tells you how much of a difference your medication makes. So apparently, being free to analyze how you want gives you enormous control over your results!

The most concerning consequence of this statistical freedom is that researchers may choose the statistical analysis most favorable to them, arbitrarily producing statistically significant results by playing with the data until something emerges. Simulation suggests that false positive rates can jump to over 50\% for a given dataset just by letting researchers try different statistical analyses until one works.53

Medical researchers have devised ways of preventing this. Researchers are often required to draft a clinical trial protocol, explaining how the data will be collected and analyzed. Since the protocol is drafted before the researchers see any data, they can’t possibly craft their analysis to be most favorable to them. Unfortunately, many studies depart from their protocols and perform different analysis, allowing for researcher bias to creep in.15, 14 Many other scientific fields have no protocol publication requirement at all.

The proliferation of statistical techniques has given us many useful tools, but it seems they have been put to use as blunt objects. One must simply beat the data until it confesses.