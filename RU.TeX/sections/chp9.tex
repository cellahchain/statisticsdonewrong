%% Chapter 9 %%%
\chapter{Ошибки делают все}
\label{chp9}

До сих пор, я предполагал, что ученые способны выполнять статистические вычисления с идеальной точностью и ошибаться могут только в выборе подходящих для этих вычислений цифр. Ученые могут неправильно использовать результаты статистического анализа или не в состоянии выполнить соответствующие расчеты, но они ведь могут, как минимум, правильно рассчитать \emph{p-}значение?

Возможно, нет. 

Обзоры статистически значимых результатов, представленных в медицинских и психологических исследованиях, показывают, что многие \emph{p-}значения ошибочны, и некоторые статистически незначимые результаты на самом деле значимы, если правильно их пересчитать. \cite{gotzsche_believability_2006,bakker_misreporting_2011} Другие обзоры находят примеры неверной классификации данных, ошибочного дублирования данных, использование полностью неверных наборов данных в анализ и других ошибок, - все спрятаны в статьях, которые не содержат описания проведённого анализа, достаточно подробного для того, чтобы эти ошибки можно было легко заметить. \cite{baggerly_deriving_2009,gotzsche_methodology_1989} 

Солнечный свет - лучшее средство дезинфекции, и многие ученые призывали к тому, что экспериментальные данные должны быть доступны в интернете. В некоторых областях, это стало распространенной практикой: существуют базы данных генных последовательностей, белковых структур, астрономических наблюдений и коллекции данных земных наблюдений, содержащие вклад тысяч различных ученых. Многие другие сферы науки, однако, не могут поделиться своими данными ввиду непрактичности (данные физики элементарных частиц могут содержать терабайты информации), невозможности разглашения (медицинские исследования), отсутствия финансирования или технической поддержки или просто исходя из желания сохранить контроль над данными и всеми открытиями, появляющиеся в результате их анализа. И даже если бы все данные были доступны, стал бы кто-нибудь их анализировать с целью поиска ошибок?   

Схожим образом, ученые в некоторых областях стали делать свой статистический анализ общедоступным путём использования умных технологических инструментов. Например, инструмент под названием Sweave позволяет легко встраивать статистический анализ, сделанный с исопльзованием популярного языка программирования R, в научные статьи, написанные с помощью \LaTeX, считающийся стандартом для написания научных и математических публикаций. Результат выглядит точно также, как и любая научная статья, но другой ученый, прочитавший публикацию и заинтересовавшийся используемыми методами, может скачать исходный код, в котором показано, каким образом были проведены все расчеты. Но будут ли ученые пользоваться такой возможностью? Никто же не достигает научной славы, проверяя код на наличие опечаток.

Другим решением может выступать повторение исследования. Если ученые тщательно воссоздадут ход эксперимента других ученых и подтвердят их результаты, будет намного проще исключить возможность опечатки, которая может привести к ошибочным результатам. Повторение также устраняет случайные ложно положительные результаты. Многие ученые утверждают, что экспериментальное повторение - это основа науки, поскольку ни одна новая идея не принимается до тех пор, пока она не была независимо проверена и перепроверена учеными по всему миру и оказалась способной выдерживать критику.  

Это не совсем верно: ученые часто принимают результаты предыдущих исследований на веру, хотя иногда решают методично перепроверять предыдущие работы. Например, один новый проект ставит своей целью воспроизвести результаты исследований из крупных психологических журналов, чтобы определить, какое количество статей выдерживают проверку временем и по каким характеристикам можно предсказать, насколько статья способна выдержать последующие перепроверки.\footnote{\href{http://openscienceframework.org/reproducibility/}{The Reproducibility Project}} 

В другом примере, исследователи рака из компании Amgen провели повторно 53 выдающихся доклинических исследования рака. (``Доклинические'' исследования в данном случае подразумеваются исследования, в которых не участвовали пациенты, поскольку в них проверялись новые и неподтвержденные идеи.) Несмотря на сотрудничество с авторами оригинальных статей, исследователям из Amgen удалось повоторить только шесть исследований. \cite{begley_drug_2012} Исследователи из компании Bayer сообщали о таких же проблемах в процессе тестирования возможных новых лекарств, обнаруженных в опубликованных статьях. \cite{prinz_believe_2011}

Это тревожит. Прослеживается ли эта тенденция во всех видах медицинских исследований? По-видимому, да: из списка наиболее цитируемых статей по медицине, четверть оказались непроверенными после публикации, а треть работ содержали преувеличенные или ошибочные результаты, как показали последующие исследования. \cite{ioannidis_contradicted_2005} Это не настолько экстремальные результаты, как у исследователей из Amgen, но заставляет задуматься, какие еще серьёзные ошибки остаются незамеченными в важных иследованиях. Повторение исследования - всё еще не самая распространенная практика, как, возможно, нам бы хотелось, и результаты ее не всегда приятны.

