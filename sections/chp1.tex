%%% Prologue %%%
\chapter*{Вступление}
\label{chp0}
\addcontentsline{toc}{chapter}{Вступление}
Если вы - учёный, возможно, в своей деятельности используете статистику для анализа данных. При поиске ответов на научные проблемы мы полагаемся на статистику, начиная с базовых t-тестов или подсчёта стандартной ошибки и заканчивая регрессией Кокса и методом подбора контрольной группы по индексу соответствия. %перевод термина - https://ru.wikipedia.org/wiki/%DD%EA%EE%ED%EE%EC%E5%F2%F0%E8%F7%E5%F1%EA%E8%E5_%EC%E5%F2%EE%E4%FB_%E2_%EE%F6%E5%ED%E8%E2%E0%ED%E8%E8#.D0.9C.D0.B5.D1.82.D0.BE.D0.B4_.D0.BF.D0.BE.D0.B4.D0.B1.D0.BE.D1.80.D0.B0_.D0.BA.D0.BE.D0.BD.D1.82.D1.80.D0.BE.D0.BB.D1.8C.D0.BD.D0.BE.D0.B9_.D0.B3.D1.80.D1.83.D0.BF.D0.BF.D1.8B_.D0.BF.D0.BE_.D0.B8.D0.BD.D0.B4.D0.B5.D0.BA.D1.81.D1.83_.D1.81.D0.BE.D0.BE.D1.82.D0.B2.D0.B5.D1.82.D1.81.D1.82.D0.B2.D0.B8.D1.8F

Это прискорбно, поскольку большинство из нас просто не умеет делать статистические подсчёты.

\emph{Неправильная статистика} - это руководство, описывающее популярные статистические ошибки и просчёты, которые совершают учёные ежедневно в лабораториях или в рецензируемых журналах. Многие из этих ошибок настолько распространены в научной литературе, что ставят под сомнение результаты большого количества статей.

Эта книга не предполагает наличие у читателя никаких статистических знаний: вы можете прочитать её перед своим первым курсом по статистике или после 30 лет научной деятельности.

Если вы обнаружите какие-либо ошибки, опечатки или можете предложить другие популярные заблуждения по теме -  \hyperref[chp1:contact]{свяжитесь со мной}. 

В попытке создать наиболее полную коллекцию статистических ошибок, я заключил контракт на публикацию \emph{Неправильной статистики} в виде расширенной книги, включающей новые разделы о статистическом моделировании, дополнительных математических разъяснениях и многом другом. \href{http://statisticsdonewrong.com}{Здесь} можно подписаться рассылку по электронной почте, а если сайт оказался вам полезным, может быть вам также понравится \href{http://www.nostarch.com/statsdonewrong}{моя книга} - она уже опубликована.


%% Chapter 1 %%%
\chapter{Введение}
%\addcontentsline{toc}{chapter}{Введение}
\label{chp1}

В заключительной главе своей знаменитой книги \emph{"Лгать с помощью статистики"} Дэррел Хафф говорит нам, что "любые статьи, имеющие отношение к медицине" или опубликованные научными лабораториями и университетами, заслуживают нашего доверия - не безусловного, но, определённо большего, чем публикации СМИ или заявления политиков. Всё же книга Хаффа была посвящена вводящим в заблуждение статистическим трюкам и ухищрениям, так часто используемыми в политике и СМИ, но мало кто выражает недовольство о качестве статистики, выполняемой профессиональными учеными. Ученые ищут ответы на вопросы, а не защиту от политических оппонентов.


Статистический анализ данных - это основа науки. Откройте любую страницу вашего любимого медицинского журнала и удивитесь популярности статистики: \emph{t} тесты, \emph{p}-значения, модели пропорциональных рисков, относительные риски, логистические регрессии, методы наименьших квадратов и доверительные интервалы. Статистики дали учёным функциональные инструменты для организации и анализа сложных наборов данных, и учёные их с удовольствием приняли.
 
Не приняли они, однако, сопутствующее статистическое \emph{образование}, и многие бакалаврские программы до сих пор не требуют вообще никакой статистической подготовки.

Начиная с 1980-х годов, исследователи описали множество статистических ошибок и заблуждений, встречающихся в популярной рецензируемой научной литературе, и обнаружили, что многие научные статьи - возможно, даже большинство, - стали жертвами этих ошибок. Недостаточная статистическая мощность отражает многие исследования, неспособные найти искомое; множественные сравнения и неверно интерпретированные \emph{p}-значения приводят к многочисленным ошибкам первого рода; гибкий анализ данных с лёкостью позволяет найти корреляцию там, где её не существует. Проблема не в подделывании результатов, а в низком уровне владения статистикой - настолько низком, что некоторые учёные делают вывод, что большинство опубликованных результатов исследований, скорее всего, ошибочны.\cite{ioannidis_why_2005}

Далее следует список наиболее вопиющих статистических ошибок, совершаемых регулярно во имя науки. Он не предполагает знания статистических методов, поскольку многие учёные вообще не получали формальной статистической подготовки. И имейте в виду: едва познакомившись с этими ошибками, вы будете встречать их \emph{везде}. Не пугайтесь. Это не повод отказываться от современной науки и обращаться заново к кровопусканию и лечению пиявками: это попытка улучшить науку, на которую мы полагаемся.
 

\section{Изменения}
\label{chp1:changes}

Обновлено в январе 2013: добавлен пример ошибки обоснования оценки: \hyperref[chp5:arms-baserateF]{подсчет количества владельцев оружия на основании опроса}.  

%%%%%%%%%%%%%%%%%%%%%%%%%%%%%%%%%%%%%%%%%%%%%%%
%%%%%%%%%%%%%%%%%%%%%%%%%%%%%%%%%%%%%%%%%%%%%%%%
%%%%%%%%%%%%%%%%%%%%%%%%%%%%%%%%%%%%%%%%%%%%%%%%
Обновлено в апреле 2013: более подробно рассказано о \hyperref[chp7:truthinflation]{взаимодействии преувеличения истины (truth inflation) и правилах предварительной остановки (early stopping rules)}, \hyperref[chp5:redherrings]{свободе исследователя в нейронауках}, \hyperref[chp3:powerunderpowered]{слабой статистической мощности в нейронауках}, \hyperref[chp5:controlfalserate]{контроле коэффициентов ложных обнаружений}, \hyperref[chp10]{искажениях в публикациях и слабых отчётах}, \hyperref[chp3:wrongturnred]{слабых исследованиях и правых поворотах на красный свет}, \hyperref[chp6:significantdiffmissed]{неправильном использовании доверительных интервалов},\hyperref[chp11]{
влиянии всех этих ошибок}, \hyperref[chp12]{о том, что можно сделать, чтобы спасти статистику}, и добавлены дополнительные ссылки и подробности в разных частях книги.

%%%%%%%%%%%%%%%%%%%%%%%%%%%%%%%%%%%%%
%%%%%%%%%%%%%%%%%%%%%%%%%%%%%%%%%%%%%
%%%%%%%%%%%%%%%%%%%%%%%%%%%%%%%%%%%%%

\section{Контакты}
\label{chp1:contact}

Я старался изо всех сил, но эта рукопись неизбежно содержит ошибки и упущения. Если у вас возникли вопросы, вы заметили ошибку или знаете что-то, что упустил я, - \href{mailto:alex@refsmmat.com}{напишите мне}\footnote{Если заметили неточности или ошибки в переводе - напишите \href{https://github.com/cellahchain/statisticsdonewrong}{переводчику.}}.

\section{Благодарности}
\label{chp1:acknow}

Спасибо д-ру Джеймсу Скотту (James Scott), чей курс по статистике дал мне основы, необходимые для написания этого текста; Мэттью Уотсону (Matthew Watson) и CharonY за неоценимую обратную связь и предложения в процессе написания этого текста; моим родителям - за отзывы и предложения; д-ру Бренту Айверсону (Brent Iverson), семинары которого побудили меня заинтересоваться злоупотреблением статистикой; и всем тем учёным и статистикам, которые, совершая ошибки, дали мне повод об этом написать.

Любые ошибки в объяснениях - мои собственные, или неточности перевода.

\section{Лицензирование}
\label{chp1:copyleft}

Эта работа предоставлена под лицензией Creative Commons Attribution 3.0 Unported \href{http://creativecommons.org/licenses/by/3.0/}{License}.

Вы можете свободно печатать, копировать, переводить, переписывать, накладывать на музыку и, в целом, делать всё что угодно с этой работой при условии, что вы ссылаетесь на меня, Alex Reinhart, и указываете ссылку на \href{http://statisticsdonewrong.com}{сайт}. (Если вы сделаете перевод, пожалуйста, сообщите мне! Я с удовольствием размещу ссылку на ваш перевод.) Подробнее о лицензии можно узнать по ссылке выше.

Используемые в тексте картинки xkcd доступны под лицензией Creative Commons Attribution NonCommercial 2.5 \href{http://creativecommons.org/licenses/by-nc/2.5/}{License} и не могут использоваться в коммерческих целях без разрешения их авторов. Подробнее \href{http://xkcd.com/license.html}{здесь}.