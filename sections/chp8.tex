%% Chapter 8 %%%
\chapter{Свобода исследователя: хорошие сомнения? }
\label{chp8}

Существует распространенное заблуждение о том, что статистика скучна и однообразна. Соберите как можно больше данных, загрузите цифры в $Excel$, $SPSS$ или $R$, и пинайте программу до тех пор, пока она не выдаст какие-нибудь красочные графики или диаграммы. Готово! Все, что остается статистикам, - это считывать результаты.

Однако, придётся выбирать, какие методы и команды использовать. Два исследователя, ищущих ответы на один и тот же вопрос, могут использовать совершенно разные методы статистического анализа. Нужно принимать множество решений:


\begin{enumerate}
	\item Какие переменные мне нужно учитывать? В медицинских испытаниях, например, стоит контролировать возраст пациентов, пол, вест, ИМТ, предыдущую медицинскую историю, курение, использование наркотиков или результаты медицинских тестов, проведённых до начала исследования. Какие из этих факторов важны, а какие можно игнорировать?

	\item Какие случаи можно исключить? Если я тестирую планы питания, возможно, я хочу исключить испытуемых, у которых в процессе исследования появилась неконтроллируемая диарея, так как их результаты будут выходить за рамки нормы.

	\item Что делать с выбросами? По неизвестным причинам, среди результатов всегда будут некоторые выпадающие из нормы, и, возможно, мне придется их исключить или анализировать отдельно. Какие случаи тогда можно считать выбросами и что мне с ними делать?

	\item Как мне определить группы? Например, я, возможно, хочу разделить испытуемых на группы ``нормальных'', ``с избыточным весом'' и ``с недостаточным весом''. Как определить границы групп? Что мне делать с мускулистыми бодибилдером, чей ИМТ находится в диапазоне значений группы ``с избыточным весом''?

	\item Что насчёт отсутствующих данных? Допустим, я пытаюсь оценить уменьшение распространения рака путем тестирования нового лекарства. Я провожу испытания в течении пяти лет, но у некоторых пациентов опухоли возвращаются через шесть или восемь лет, а в мои данные этого не будут содержать. Как мне это учитывать при оценке эффективности лекарства? 

	\item Сколько данных я должен собрать? Стоит ли мне остановиться, как только я получу определённый результат, или следует продолжать до тех пор, пока я не соберу все данные, как было запланировано?

	\item Как мне оценить полученные результаты? Действие лекарства можно оценить с помощью индивидуальных опросов пациентов, результатов испытаний, преобладания определенного симптома или, например, продолжительностью болезни.

\end{enumerate}

Прежде, чем появятся результаты, можно потратить много времени на определение того, какие методы будут наиболее уместны в данной ситуации. Научные статьи обычно содержат объяснение того, какие методы статистической обработки были использованы, но они практически никогда не объясняют причину выбора именно этого метода по сравнению с другими или какие результаты могли получить исследователи, используя другие методы. Исследователи вольны выбирать методы, которые они считают подходящими, - и, хотя они могли сделать правильный выбор, что было бы, анализируй они эти данные по-другому?    

При моделировании можно получить различающиеся в два раза размеры эффекта просто путем регулирования различных переменных, исключения определенных наборов случаев из анализа и обработки выбросов другим способом.\cite{ioannidis_why_2008} Размер эффекта - это та самая важная цифра, которая показывает, сколько различий привносит ваше лекарство. Таким образом, очевидно, что свобода выбирать, как анализировать свои данные, даёт вам немалый контроль над получаемыми результатами!  

Наиболее важное следствие такой статистической свободы заключается в том, что исследователи могут выбирать любимые методы анализа, произвольно получая статистически значимые результаты, манипулируя и играя с данными до тех пор, пока что-нибудь не появится. Моделирование показывает, что оценки ложно положительных результатов могут вырасти до 50\% для конкретного набора данных просто позволяя исследователям использовать разные статистические методы до тех пор, пока какой-нибудь не сработает.\cite{simmons_false-positive_2011}

Медицинские исследователи разработали способ предотвращения такой ситуации. От исследователей обычно требуют составить план протокола клинических испытаний, в котором бы объяснялось каким образом данные будут собираться и анализироваться. Поскольку протокол составляется до того, как исследователи увидят какие либо данные, они в принципе не могут сделать анализ наиболее благоприятным для себя. К сожалению, многие исследования отступают от протоколов и проводят различные анализы данных, делая возможным появление феномена ошибки экспериментатора.\cite{chan_discrepancies_2008,chan_empirical_2004} Во многих других исследовательских сферах в принципе отсутствует требование публикации протоколов исследования.

Распростанение статистических методов дало нам множество полезных инструментов, но, похоже, их используют в качестве тупых предметов. Надо просто бить данные до тех пор, пока они не признаются.
