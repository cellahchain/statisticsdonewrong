%% Chapter 10 %%%
\chapter{Скрываем данные}
\label{chp10}

\begin{chapquote}{Эрик С. Рэймонд}
``При достаточном количестве глаз, все ошибки выплывают на поверхность.''
\end{chapquote}


Ранее мы говорили о наиболее распространенных ошибках, которые допускают учёные, и о том, что наилучший способ их обнаружить - тщательное изучение извне. Экспертная оценка, в некоторой степени этому способствует, однако у человека, проводящего оценку, недостаточно времени для того, чтобы подробно проанализировать все данные повторно и проверить все исходные коды анализа на ошибки, - рецензенты могут лишь проверить правильность выбранной методологии исследования. Иногда они замечают очевидные ошибки, но едва заметные проблемы, как правило, пропускают.\cite{schroter_what_2008}  


Поэтому многие рецензируемые журналы и профессиональные сообщества требуют от исследователей предоставлять другим исследователям доступ к своим данным по запросу. Полные наборы данных, как правило, слишком большие по размеру, чтобы их можно было напечатать на страницах журнала, поэтому авторы лишь публикуют свои результаты в статьях, а копию данных отправляют другим исследователям по запросу. Возможно, другие исследователи смогут обнаружить ошибку или незамеченную закономерность.

Теоретически, именно так это и должно происходить. В 2005 году Джелте Уичертс с коллегами из университета Амстердама решили проанализировать недавно опубликованные статьи в нескольких знаменитых журналов Американской Психологической Ассоциации, чтобы узнать об использованных в них статистических методах. Они выбрали журналы АПА отчести потому, что это сообщество требует от авторов статей согласие на предоставление собранных в рамках исследования данных другим психологам, стремящихся проверить их результаты. 

Шесть месяцев спустя они смогли получить данные только по 64 исследованиям из 249 анализируемых. Почти три четверти авторов статей так и не прислали им свои данные.\cite{wicherts_poor_2006} 

Конечно, учёные - занятые люди и, возможно, у них просто не нашлось свободного времени, чтобы собрать свои даные, подготовить пояснительные документы, описывающие значения каждой переменной и каким образом она была измерена, и так далее.

Уитчерс и коллеги решили это проверить. Они тщательно просмотрели все исследования на наличие наиболее распространенных ошибок, которые можно было бы заметить в процессе чтения статьи, таких как: противоречивые статистические результаты, неправильное применение различных статистических тестов и обычных опечаток. По крайней мере, в половине статей содержались ошибки, как правило, незначительные, однако 15\% статей содержали описание не менее одного статистически значимого результата, который был значимым исключительно из-за ошибки. 

Далее, они искали корреляцию между этими ошибками и нежеланием авторов делиться своими данными, и, как оказалось, между ними существовала четкая связь. Авторы, отказавшиеся делиться своими данными, были в большей степени склонны допускать ошибку в своей статье, и статистические обоснования их выводов были, как правило, слабее.\cite{wicherts_willingness_2011} Поскольку большинство авторов отказались предоставлять свои данные, Уитчерс не мог провести более детальный поиск статистических ошибок, которых могло оказаться гораздо больше.

Это, конечно, не доказательство того, что авторы скрывали свои данные, боясь обнаружения их ошибок, или что они вообще знали о наличии этих ошибок. Наличие корреляции не подразумевает наличие причинно-следственной связи, но в данном случае корреляция явно намекает ``внимательно посмотри сюда.''\footnote{Шутка бесстыдно украдена из альтернативного варианта комикса \href{http://xkcd.com/552/}{http://xkcd.com/552/}}



\section{Просто опустите подробности}
\label{chp10:leaveoutdetails}

Придирчивые статистики тянут вас на дно, указывая на недостатки вашей статьи? Существует одно простое решение: публикуйте как можно меньше подробностей! Они не смогут найти ошибки, если вы не скажете как вы оценивали свои данные.

Я не утверждаю всерьёз, что злые ученые делают это намерянно, хотя, возможно, некоторые делают. Чаще детали опущены просто потому, что авторы забыли включить их в статью или в виду того, что ограничения журнала на размер статьи заставили так поступить.

Исследование можно оценить, чтобы понять, что было исключено. Учёные, ведущие медицинские испытания, должны предоставлять подробные планы исследования экспертным советам по этике до начала испытаний, поэтому одна группа исследователей получила коллекцию таких планов от одного  экспертного совета. В этих планах указывалось, какие результаты исследования будут измеряны: например, можно отслеживать различные симптомы, чтобы пронаблюдать, оказывало ли на них влияние тестируемое лекарство. Затем исследователи разыскали опубликованные результаты этих планируемых исследований и посмотрели, насколько хорошо эти результаты были представлены.   

Примерно половина планируемых результатов никогда так и не появились в научных статьях рецензируемых журналов. Многие из них оказались статистически незначимыми, поэтому были просто скрыты. Другая большая часть результатов была опубликована без подробностей, что исключало возможность в дальнейшем использовать эти результаты для мета-анализа.\cite{chan_empirical_2004} 

%убрал сноску №2, не вижу смысла ее переводить %%%

Другие обзоры сталкивались с похожими проблемами. Обзор клинических испытаний обнаружил, что большинство исследований опускают важные методологические детали, как например, \hyperref[chp7]{правила остановки} или \hyperref[chp3]{расчеты мощности}, - это свойственно, в большей степени, небольшим специализированным журналам, нежели большим общемедицинским журналам.\cite{huwiler-muntener_quality_2002}

Журналы по медицине пытаются бороться с этой проблемой путем стандартизации отчетов о результатах, таких как список \href{http://www.consort-statement.org/}{CONSORT}. Авторы должны следовать требованиям списка до момента предоставления работы на рассмотрение, а редакторы проверяют, включены ли в статью все необходимые детали. Этот список, похоже, работает: исследования, опубликованные в журналах, которые следуют рекомендациям от CONSORT, как правило, предоставляют больше существенных подробностей об исследовании, хотя и не все.\cite{plint_does_2006} К сожалению, стандарты не всегда последовательно применяются и некоторым работам удается проскочить мимо них с отсутствующими в исследовании деталями.\cite{mills_analysis_2005} Редакторам журналов придётся приложить больше усилий для соблюдения стандартов отчётности.

По-видимому, с опубликованными статьями дела обстоят не очень хорошо. А с неопубликованными исследованиями?


\section{Наука в картотеке}
\label{chp10:sciencecabinet}

Ранее мы видели влияние \hyperref[chp5:try-again]{множественных сравнений} и \hyperref[chp7:truthinflation]{преувеличения истины} на результаты исследования. Эти проблемы возникают, когда в исследованиях проводятся многочисленные сравнения с низкой статистической мощностью, повышая, тем самым, количество ложноположительных результатов и оценки размеров эффекта, и такие исследования встречаются в публикациях постоянно.

Однако, не каждое исследование публикуется. Мы всегда видим только часть медицинских исследований, например, потому что немногие учёные публикуют результаты из разряда ``Мы протестировали данное лекарство и непохоже, что оно сработало.''

Рассмотрим пример: изучение опухолевого супрессора белка TP53 и его влияния на рак головы и шеи. Ряд исследований предполагали, что измерение TP53 может быть использовано для прогнозирования показателей смертности от рака, поскольку он участвует в регуляции роста и развития клеток, и, следовательно, должен функционировать правильно, для предотвращения рака. Когда все 18 опубликованных исследований по изучению TP53 и рака были проанализированы вместе, в результате была получена статистически значимая корреляция: измеряя TP53 можно явно определить, насколько вероятно умереть от опухоли.

Но, предположим, что мы откопали \emph{неопубликованные} результаты исследований TP53: данные, которые упоминались в других исследованиях, но не были опубликованы и проанализированы. Добавьте эти данные к уже имеющимся, и статистически значимый эффект исчезает.\cite{kyzas_selective_2005} В конце концов, немногие авторы потрудились опубликовать данные, показывающие отсутствие корреляции, поэтому в мета-анализе использовалась фактически смещенная выборка. 

Аналогичное исследование рассматривало ребоксетин - антидепрессант, продаваемый компанией Pfizer. Несколько опубликованных исследований предполагали его эффективность по сравнению с плацебо, что привело к тому, что в нескольких европейских странах оно было одобрено для лечения пациентов в состоянии депрессии. Немецкому институту Качества и Эффективности в Здравоохранении, ответственному за оценку медицинских лекарств, удалось получить неопубликованные данные клинических испытаний от Pfizer - в три раза больше данных, чем когда-либо опубликованных - и тщательно их проанализировать. В результате, ребоксетин оказался неэффективным. Pfizer просто убедил общественность в том, что антидепрессант эффективен, пренебрегая упоминанием исследований, которые доказывали обратное.\cite{eyding_reboxetine_2010}

Эта проблема широко известна под названием ``ошибка публикации'' или ``проблема картотечного ящика'': многие исследования остаются неопубликованными, ``хранятся в ящиках'' в течении многих лет, несмотря на ценные данные, которые в них содержатся.

Проблема заключается не просто в смещении на опубликованные результаты. Неопубликованные исследования ведут к дублированию усилий - если другие учёные не знают, что вы провели исследование, они вполне могут снова его провести, тратя напрасно деньги и усилия. 

Регуляторы и научные журналы пытались решить эту проблему. Управление по санитарному надзору за качеством пищевых продуктов и медикаментов США (FDA) требует, чтобы определённые виды клинических испытаний были зарегистрированы на их веб-сайте \href{https://clinicaltrials.gov/}{ClinicalTrials.gov} до начала испытаний, а также, чтобы результаты этих испытаний были опубликованы в течении года после окончания испытаний. Точно также, Международная комиссия редакторов медицинских журналов в 2005 году объявила, что они не будут публиковать исследования, которые не были предварительно зарегистрированы.

К сожалению, обзор 738 зарегистрированных клинических испытаний обнаружил, что только 22\% соответствовали законным требованиям к публикации.\cite{prayle_compliance_2012} Управление по санитарному надзору (FDA) не оштрафовала ни одну фармацевтическую компанию за несоблюдение их требований, а журналы всё еще не требуют предварительной регистрации испытаний. И большинство исследований просто исчезают.
